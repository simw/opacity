\section{Gyulassy/opacity3/src/main.cpp File Reference}
\label{main_8cpp}\index{Gyulassy/opacity3/src/main.cpp@{Gyulassy/opacity3/src/main.cpp}}
{\tt \#include $<$iostream$>$}\par
{\tt \#include $<$string$>$}\par
{\tt \#include \char`\"{}progressbar.h\char`\"{}}\par
{\tt \#include \char`\"{}timer.h\char`\"{}}\par
{\tt \#include \char`\"{}driver.h\char`\"{}}\par
{\tt \#include \char`\"{}glv1/radcalcerwrapper.h\char`\"{}}\par
{\tt \#include \char`\"{}glv3/glvradiative3.h\char`\"{}}\par
{\tt \#include \char`\"{}store2d/store.h\char`\"{}}\par
{\tt \#include \char`\"{}./glv3/qperpcalculator1.h\char`\"{}}\par
{\tt \#include \char`\"{}./glv3/qperpcalculator3.h\char`\"{}}\par
{\tt \#include \char`\"{}./glv3/qperpgenerator3.h\char`\"{}}\par
\subsection*{Defines}
\begin{CompactItemize}
\item 
\#define {\bf NO\_\-INPUT}
\end{CompactItemize}
\subsection*{Functions}
\begin{CompactItemize}
\item 
void {\bf GetInput} (std::string \&resume, std::string \&inputFile, std::string \&outputFile, long \&NumberOfIterations)
\item 
int {\bf main} (int argc, char $\ast$argv[$\,$])
\end{CompactItemize}


\subsection{Define Documentation}
\index{main.cpp@{main.cpp}!NO_INPUT@{NO\_\-INPUT}}
\index{NO_INPUT@{NO\_\-INPUT}!main.cpp@{main.cpp}}
\subsubsection{\setlength{\rightskip}{0pt plus 5cm}\#define NO\_\-INPUT}\label{main_8cpp_d62650a8615e3aa9191d667f58c439f0}




Definition at line 24 of file main.cpp.

\subsection{Function Documentation}
\index{main.cpp@{main.cpp}!GetInput@{GetInput}}
\index{GetInput@{GetInput}!main.cpp@{main.cpp}}
\subsubsection{\setlength{\rightskip}{0pt plus 5cm}void GetInput (std::string \& {\em resume}, std::string \& {\em inputFile}, std::string \& {\em outputFile}, long \& {\em NumberOfIterations})}\label{main_8cpp_255d3fd88277d4303927db8316098dc7}


Get input from the user in order to run a Glv calculation If we have define NO\_\-INPUT above then then the user isn't asked, it just runs with a default set of parameters 

Definition at line 168 of file main.cpp.

Referenced by main().

\begin{Code}\begin{verbatim}170 {
171   // Get required input for the Monte Carlo run
172   // First, ask whether we are resuming the Monte Carlo statistics from a previous run
173   do
174   {
175     std::cout << "Resume from a previous run? (y/n) ";
176 #ifdef NO_INPUT
177    resume = "n";
178    std::cout << resume;
179 #else
180     std::cin >> resume;
181 #endif
182     std::cout << "\n";
183   } 
184   while ( !( resume=="y" || resume=="n" ) );
185   
186   // Second, filenames
187   std::cout << "\nEnter input file and output file\n";
188   std::cout << "(no spaces in file names, separate two names with a space): ";
189 #ifdef NO_INPUT
190   inputFile = "./results/temp.params"; outputFile = "./results/temp.txt";
191   std::cout << std::endl << inputFile << " " << outputFile;
192 #else
193   std::cin >> inputFile >> outputFile;
194 #endif
195   std::cout << "\n";
196   
197   // Finally (thirdly), how many Monte Carlo iterations do we want?
198   std::cout << "How many Monte Carlo iterations? ";
199 #ifdef NO_INPUT
200   NumberOfIterations = 100000;
201   std::cout << NumberOfIterations;
202 #else
203   std::cin >> NumberOfIterations;
204 #endif
205   std::cout << std::endl << std::endl;
206 }
\end{verbatim}
\end{Code}


\index{main.cpp@{main.cpp}!main@{main}}
\index{main@{main}!main.cpp@{main.cpp}}
\subsubsection{\setlength{\rightskip}{0pt plus 5cm}int main (int {\em argc}, char $\ast$ {\em argv}[$\,$])}\label{main_8cpp_0ddf1224851353fc92bfbff6f499fa97}




Definition at line 64 of file main.cpp.

References GetInput(), Timer::GetResult(), ProgressBar::PrintFinal(), ProgressBar::PrintPreliminaries(), ProgressBar::PrintProgress(), ProgressBar::SetNow(), Timer::StartTimer(), and Timer::StopTimer().

\begin{Code}\begin{verbatim}65 {
66   // Sample program doing a Monte Carlo calculation
67   // We need to choose the dimensionality of the Monte Carlo at COMPILE time
68   // This is in the NUMRANDOMS macros above
69   // Note that the calculation engine (eg GLV) may use more than one random number
70   // per eg order in opacity, so NUMRANDOMS is a multiple of opacity
71 
72   std::cout << std::endl;
73   std::cout << "Monte Carlo: Opacity version 3 by Simon Wicks" << std::endl;
74   std::cout << "(using " << NUMRANDOMS << " random numbers)." << std::endl << std::endl;
75 
76   // The only input needed from the user is:
77   // 1) whether we are resuming from a previous run,
78   // 2) the filename of the parameters (+data if resuming) to read in
79   // 3) the filename to which to write our results
80   // 4) the number of iterations to do
81   long NumberOfIterations;
82   std::string resume = ""; std::string inputFile = ""; std::string outputFile = "";
83   GetInput( resume, inputFile, outputFile, NumberOfIterations );
84 
85   // The templatized nature of the Driver means that a number of different calculations
86   // can be done just by changing the code here in main.
87   // A number of examples are given below
88   // Just get rid of the /* */ around the block
89 
90   /*
91   // Glv calculation using most of the old code ('version 1'), but using a wrapper
92   // to put it into the new templatized Driver
93   // GlvCalcer only really does a dq dphi_q integration and throws away an extra
94   // random number per dimension, hence it uses 3 random number per order in opacity
95   // ie opacity = NUMRANDOMS/3
96   // TODO: change this to work again
97   //Driver<RadCalcerWrapper<9>, Store2D, 9> myDriver;
98   */
99 
100   /*
101   // Glv calculation, using the new ('version 3') calculation code (GlvRadiative3),
102   // but using a wrapper around the old ('version 1') QperpCalculator code (QperpCalculator1)
103   // (which is slower than the newer version, but is a good test to give the same answer)
104   // GlvRadiative3 does a dq dphi_q integration, hence opacity = NUMRANDOMS/2
105   Driver<GlvRadiative3
106     <QperpGenerator3<NUMRANDOMS/2>, QperpCalculator1<NUMRANDOMS/2>, NUMRANDOMS>, 
107     Store2D, NUMRANDOMS> myDriver;
108   */
109   
110   // Glv calculation using the new calculation code (GlvRadiative3) and the
111   // new qperpcalculator code (QperpCalculator3)
112   // GlvRadiative3 does a dq dphi_q integration, hence opacity = NUMRANDOMS/2
113   Driver<GlvRadiative3
114     <QperpGenerator3<NUMRANDOMS/2>, QperpCalculator3<NUMRANDOMS/2>, NUMRANDOMS>, 
115     Store2D, NUMRANDOMS> myDriver;
116 
117   /*
118   // TODO: write a GlvRadiative4 which includes the z integration
119   */
120 
121 
122   // End of different calculation examples
123 
124   // Call the driver to set up itself and all the other objects
125   // If ok, it'll return 0. If not, then it'll return something else.
126   int errCode = myDriver.Setup( resume, inputFile );
127   if ( errCode != 0 )
128   {
129     std::cerr << "Error on setting up, code = " << errCode << std::endl;
130     return EXIT_SUCCESS;
131   }
132 
133   // Everything is ready in the calculation, just set up the final
134   // bits for the user interface: timer and progress bar
135   // ProgressBar: The first number of the current point (start = 0),
136   // the second is the end point, the total number
137   Timer timer;
138   ProgressBar progBar(0,NumberOfIterations);
139   progBar.PrintPreliminaries();
140   
141   // And now we run the main loop
142   timer.StartTimer();
143   for (long i=0; i<NumberOfIterations; ++i)
144   {
145     progBar.SetNow(i);
146     progBar.PrintProgress();
147     myDriver.RunOneIteration();
148   }
149   progBar.SetNow(NumberOfIterations);
150   progBar.PrintProgress();
151   timer.StopTimer();
152   // And the main loop is done!
153 
154   // Tell the driver to save our results to file
155   myDriver.SaveResults( outputFile );
156  
157   // Print to screen a few details of the run
158   progBar.PrintFinal();
159   double totTime = timer.GetResult();
160   std::cout << "Total time = " << totTime << " seconds\n";
161   std::cout << "Time per iteration = " << totTime/static_cast<double>(NumberOfIterations);
162   std::cout << "\n";
163   std::cout << std::endl;
164 
165   return EXIT_SUCCESS;
166 }
\end{verbatim}
\end{Code}


