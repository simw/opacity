\section{SwUtils Namespace Reference}
\label{namespaceSwUtils}\index{SwUtils@{SwUtils}}


\subsection*{Functions}
\begin{CompactItemize}
\item 
long {\bf \_\-FindNextPowerOfTwo} (long value)
\item 
long {\bf \_\-Combinatoric} (long n, long r)
\item 
long {\bf \_\-factorial} (long value, long endval=1)
\item 
long {\bf power} (long num, long pow)
\item 
bool {\bf \_\-TestBitI} (unsigned long num, unsigned long i)
\item 
void {\bf \_\-NumberToBoolArray} (unsigned long num, std::vector$<$ bool $>$ \&boolarray, unsigned long length)
\item 
long {\bf \_\-CountNumZeroes} (unsigned long num, unsigned long length)
\item 
bool {\bf \_\-IsPowerOfTwo} (long value)
\end{CompactItemize}
\subsection*{Variables}
\begin{CompactItemize}
\item 
const double {\bf pi} = 3.141592653589793238
\item 
const double {\bf overpi} = 0.3183098861837906715
\item 
const double {\bf hbarc} = 0.197
\end{CompactItemize}


\subsection{Function Documentation}
\index{SwUtils@{SwUtils}!_Combinatoric@{\_\-Combinatoric}}
\index{_Combinatoric@{\_\-Combinatoric}!SwUtils@{SwUtils}}
\subsubsection{\setlength{\rightskip}{0pt plus 5cm}long SwUtils::\_\-Combinatoric (long {\em n}, long {\em r})}\label{namespaceSwUtils_0183044c7c4da9b92fd4a0d94c5632f3}




Definition at line 22 of file constants.cpp.

References \_\-factorial().

Referenced by RadCalcer::GetdNdk2dx().

\begin{Code}\begin{verbatim}23 {
24   // This returns the combinatoric factor n C r
25   // ie n! / ( (n-r)! r! )
26   // Note: sum of n C r over r gives 2^n
27 
28   if ( r >= n || r <= 0 )
29     return 1;
30 
31   // _factorial (n,b) = n!/(b-1)! ie n*(n-1)*...*b
32   return ( _factorial( n, n-r+1 ) / _factorial( r ) );
33 }
\end{verbatim}
\end{Code}


\index{SwUtils@{SwUtils}!_CountNumZeroes@{\_\-CountNumZeroes}}
\index{_CountNumZeroes@{\_\-CountNumZeroes}!SwUtils@{SwUtils}}
\subsubsection{\setlength{\rightskip}{0pt plus 5cm}long SwUtils::\_\-CountNumZeroes (unsigned long {\em num}, unsigned long {\em length})\hspace{0.3cm}{\tt  [inline]}}\label{namespaceSwUtils_3b26ae74b9091b6bf31b181cb245c7c1}




Definition at line 51 of file constants.h.

References \_\-TestBitI().

\begin{Code}\begin{verbatim}52 {
53   long tot = 0;
54   for (unsigned long i=0; i!=length; ++i)
55   {
56     if ( _TestBitI( num, i ) )
57       ++tot;
58   }
59   return tot;
60 }
\end{verbatim}
\end{Code}


\index{SwUtils@{SwUtils}!_factorial@{\_\-factorial}}
\index{_factorial@{\_\-factorial}!SwUtils@{SwUtils}}
\subsubsection{\setlength{\rightskip}{0pt plus 5cm}long SwUtils::\_\-factorial (long {\em value}, long {\em endval} = {\tt 1})\hspace{0.3cm}{\tt  [inline]}}\label{namespaceSwUtils_3bf3423904e7b04ce19649b33bb7de77}




Definition at line 14 of file constants.h.

Referenced by \_\-Combinatoric(), ZposGenerator::FindRandomPositions(), and GlvRadiative3$<$ TqperpGenerate, TqperpCalculate, numOfRandoms $>$::GetdNdxdk().

\begin{Code}\begin{verbatim}15 {
16   if (value > endval)
17     return ( value * _factorial( value - 1 ) );
18 
19   return endval;
20 }
\end{verbatim}
\end{Code}


\index{SwUtils@{SwUtils}!_FindNextPowerOfTwo@{\_\-FindNextPowerOfTwo}}
\index{_FindNextPowerOfTwo@{\_\-FindNextPowerOfTwo}!SwUtils@{SwUtils}}
\subsubsection{\setlength{\rightskip}{0pt plus 5cm}long SwUtils::\_\-FindNextPowerOfTwo (long {\em value})}\label{namespaceSwUtils_26183064bbbb36f547635316aec1dd97}




Definition at line 7 of file constants.cpp.

References \_\-IsPowerOfTwo().

Referenced by StatGathering::ConvergenceTable::AddOneSetOfResults(), and StatGathering::ConvergenceTable::SetResultsSoFar().

\begin{Code}\begin{verbatim}8 {
9   long myreturn = 0;
10   --value;
11   do
12   {
13     ++value;
14     if ( _IsPowerOfTwo( value ) )
15       myreturn = value;
16   }
17   while( myreturn == 0 );
18 
19   return myreturn;
20 }
\end{verbatim}
\end{Code}


\index{SwUtils@{SwUtils}!_IsPowerOfTwo@{\_\-IsPowerOfTwo}}
\index{_IsPowerOfTwo@{\_\-IsPowerOfTwo}!SwUtils@{SwUtils}}
\subsubsection{\setlength{\rightskip}{0pt plus 5cm}bool SwUtils::\_\-IsPowerOfTwo (long {\em value})\hspace{0.3cm}{\tt  [inline]}}\label{namespaceSwUtils_a958298ae1d796946e5784f3339ffba1}




Definition at line 62 of file constants.h.

Referenced by \_\-FindNextPowerOfTwo().

\begin{Code}\begin{verbatim}63 {
64   if (value < 1)
65     return false;
66 
67   return (value & (~value+1)) == value;  //~value+1 equals a two's complement -value
68 }
\end{verbatim}
\end{Code}


\index{SwUtils@{SwUtils}!_NumberToBoolArray@{\_\-NumberToBoolArray}}
\index{_NumberToBoolArray@{\_\-NumberToBoolArray}!SwUtils@{SwUtils}}
\subsubsection{\setlength{\rightskip}{0pt plus 5cm}void SwUtils::\_\-NumberToBoolArray (unsigned long {\em num}, std::vector$<$ bool $>$ \& {\em boolarray}, unsigned long {\em length})\hspace{0.3cm}{\tt  [inline]}}\label{namespaceSwUtils_1b9e53b0f01225242c82fe97e878bd6d}




Definition at line 43 of file constants.h.

References \_\-TestBitI().

Referenced by \_\-FillAllRows1D(), \_\-FillAllRows2DSymmetric(), \_\-Iterate1DForZeroes(), \_\-Iterate2DForZeroes(), QperpGenerator3$<$ n $>$::GetQsThetas(), and QperpGenerator::SetZeroedQs().

\begin{Code}\begin{verbatim}44 {
45   for (unsigned long i=0; i!=length; ++i)
46   {
47     boolarray[(length-1)-i] = _TestBitI( num, i );
48   }
49 }
\end{verbatim}
\end{Code}


\index{SwUtils@{SwUtils}!_TestBitI@{\_\-TestBitI}}
\index{_TestBitI@{\_\-TestBitI}!SwUtils@{SwUtils}}
\subsubsection{\setlength{\rightskip}{0pt plus 5cm}bool SwUtils::\_\-TestBitI (unsigned long {\em num}, unsigned long {\em i})\hspace{0.3cm}{\tt  [inline]}}\label{namespaceSwUtils_58cbacdf24d6a8eddaae9fbbd58ecf78}




Definition at line 34 of file constants.h.

Referenced by \_\-CountNumZeroes(), \_\-NumberToBoolArray(), and QperpCalculator3$<$ n $>$::SetQsThetas().

\begin{Code}\begin{verbatim}35 {
36   int bit = ((num >> i) & 1);
37   if (bit == 1)
38     return true;
39     
40   return false;
41 }
\end{verbatim}
\end{Code}


\index{SwUtils@{SwUtils}!power@{power}}
\index{power@{power}!SwUtils@{SwUtils}}
\subsubsection{\setlength{\rightskip}{0pt plus 5cm}long SwUtils::power (long {\em num}, long {\em pow})\hspace{0.3cm}{\tt  [inline]}}\label{namespaceSwUtils_3974619c250bd02a70c9cfcd88384223}




Definition at line 25 of file constants.h.

Referenced by RadCalcer::GetdNdk2dx(), and QperpCalculator::QperpCalculator().

\begin{Code}\begin{verbatim}26 {
27   long res = num;
28   for (long i=1; i!=pow; ++i)
29     res *= num;
30 
31   return res;
32 }
\end{verbatim}
\end{Code}




\subsection{Variable Documentation}
\index{SwUtils@{SwUtils}!hbarc@{hbarc}}
\index{hbarc@{hbarc}!SwUtils@{SwUtils}}
\subsubsection{\setlength{\rightskip}{0pt plus 5cm}const double {\bf SwUtils::hbarc} = 0.197}\label{namespaceSwUtils_d2fad768d402d77bfb40f4c87adb8966}




Definition at line 11 of file constants.h.

Referenced by RadCalcer::SetXonly().\index{SwUtils@{SwUtils}!overpi@{overpi}}
\index{overpi@{overpi}!SwUtils@{SwUtils}}
\subsubsection{\setlength{\rightskip}{0pt plus 5cm}const double {\bf SwUtils::overpi} = 0.3183098861837906715}\label{namespaceSwUtils_b08cebc8959f71a8ae78b8d56c6d6be6}




Definition at line 10 of file constants.h.\index{SwUtils@{SwUtils}!pi@{pi}}
\index{pi@{pi}!SwUtils@{SwUtils}}
\subsubsection{\setlength{\rightskip}{0pt plus 5cm}const double {\bf SwUtils::pi} = 3.141592653589793238}\label{namespaceSwUtils_3e747e4d95465902347281e3cc1bcb0d}




Definition at line 9 of file constants.h.