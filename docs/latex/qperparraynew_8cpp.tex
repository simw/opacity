\section{Gyulassy/opacity3/src/glv1/qperparraynew.cpp File Reference}
\label{qperparraynew_8cpp}\index{Gyulassy/opacity3/src/glv1/qperparraynew.cpp@{Gyulassy/opacity3/src/glv1/qperparraynew.cpp}}
{\tt \#include \char`\"{}qperparraynew.h\char`\"{}}\par
{\tt \#include \char`\"{}../constants.h\char`\"{}}\par
\subsection*{Functions}
\begin{CompactItemize}
\item 
void {\bf \_\-FillFirstRow} (unsigned long n, unsigned long z, std::vector$<$ bool $>$ zeroed, array2D \&qiqjs, array3D \&sumqiqjs)
\item 
void {\bf \_\-FillNextRow} (unsigned long n, unsigned long i, unsigned long z, std::vector$<$ bool $>$ zeroed, array2D \&qiqjs, array3D \&sumqiqjs)
\item 
void {\bf \_\-FillAllRows2DSymmetric} (unsigned long n, array2D \&qiqjs, array3D \&sumqiqjs, long z)
\item 
void {\bf \_\-FillAllRows1D} (unsigned long n, array1D \&khatqis, array2D \&sumkhatqis, long z)
\item 
void {\bf \_\-Fill2DFrom1D} (unsigned long n, array2D \&sumkhatqi\_\-iton, array3D \&sumkhatqikhatqj\_\-itonjton, long z)
\item 
void {\bf \_\-DotProducts2D} (unsigned long n, MyArray \&mags, MyArray \&ths, array2D \&res)
\item 
void {\bf \_\-DotProducts1D} (unsigned long n, double mag1, double th1, MyArray \&mags, MyArray \&ths, array1D \&res)
\item 
void {\bf \_\-OuterProduct1D1D} (unsigned long n, MyArray \&mags1, MyArray \&ths1, MyArray \&mags2, MyArray \&ths2, array2D \&res)
\item 
void {\bf \_\-OuterProduct0D1D} (unsigned long n, double mag1, double th1, MyArray \&mags2, MyArray \&ths2, array1D \&res)
\item 
void {\bf \_\-Iterate2DForZeroes} (unsigned long n, array2D \&qiqjs, array3D \&sumqiqjs)
\item 
void {\bf \_\-Iterate1DForZeroes} (unsigned long n, array1D \&khatqis, array2D \&sumkhatqis)
\end{CompactItemize}


\subsection{Function Documentation}
\index{qperparraynew.cpp@{qperparraynew.cpp}!_DotProducts1D@{\_\-DotProducts1D}}
\index{_DotProducts1D@{\_\-DotProducts1D}!qperparraynew.cpp@{qperparraynew.cpp}}
\subsubsection{\setlength{\rightskip}{0pt plus 5cm}void @8::\_\-DotProducts1D (unsigned long {\em n}, double {\em mag1}, double {\em th1}, MyArray \& {\em mags}, MyArray \& {\em ths}, array1D \& {\em res})\hspace{0.3cm}{\tt  [static]}}\label{qperparraynew_8cpp_aea824af3e9dc625dd755fe24f0492b2}




Definition at line 147 of file qperparraynew.cpp.

Referenced by QperpCalculator::CalcQmatrices2().

\begin{Code}\begin{verbatim}149 {
150   // TODO: put in asserts to check that the dimensions of mags, ths and res are correct
151 
152   // No need to pre-calculate the cosine this time, we'll just jump in
153   for (unsigned long i=0; i!=n; ++i)
154     res[i] = mag1 * mags[i] * cos( th1 - ths[i] );
155 
156   // And we're done!
157 }
\end{verbatim}
\end{Code}


\index{qperparraynew.cpp@{qperparraynew.cpp}!_DotProducts2D@{\_\-DotProducts2D}}
\index{_DotProducts2D@{\_\-DotProducts2D}!qperparraynew.cpp@{qperparraynew.cpp}}
\subsubsection{\setlength{\rightskip}{0pt plus 5cm}void @8::\_\-DotProducts2D (unsigned long {\em n}, MyArray \& {\em mags}, MyArray \& {\em ths}, array2D \& {\em res})\hspace{0.3cm}{\tt  [static]}}\label{qperparraynew_8cpp_3b87a5052e73e5de13161681e25cf1fa}




Definition at line 111 of file qperparraynew.cpp.

Referenced by QperpCalculator::CalcQmatrices2().

\begin{Code}\begin{verbatim}112 {
113   // TODO: put in asserts to check that the dimensions of mags, ths and res are correct
114 
115   // The operations sin and cos are numerically expensive
116   // For the dot products: will need cos( ths[i] - ths[j] )
117   // = cos(ths[i]) cos(ths[j]) + sin(ths[i]) sin(ths[j])
118   // Here: evaluate them 2n times, instead of possibly n^2 or 0.5*n(n+1)
119   MyArray sines(n), cosines(n);
120   for (unsigned long i=0; i!=n; ++i)
121   {
122     sines[i] = sin( ths[i] );
123     cosines[i] = cos( ths[i] );
124   }
125 
126   // Now, find the 2D matrix qi.qj
127   // We know that it is symmetric, so first we'll evaluate the off-diagonal on
128   // one side, then mirror it over to the other side
129   for (unsigned long i=0; i!=n; ++i)
130     for (unsigned long j=i+1; j!=n; ++j)
131     {
132       res[i][j] = mags[i] * mags[j] * ( cosines[i]*cosines[j] + sines[i]*sines[j] );
133       res[j][i] = res[i][j];
134     }
135 
136   // Now all that's left is the diagonal
137   for (unsigned long i=0; i!=n; ++i)
138     res[i][i] = mags[i] * mags[i];
139 
140   // And we're done!
141 }
\end{verbatim}
\end{Code}


\index{qperparraynew.cpp@{qperparraynew.cpp}!_Fill2DFrom1D@{\_\-Fill2DFrom1D}}
\index{_Fill2DFrom1D@{\_\-Fill2DFrom1D}!qperparraynew.cpp@{qperparraynew.cpp}}
\subsubsection{\setlength{\rightskip}{0pt plus 5cm}void @8::\_\-Fill2DFrom1D (unsigned long {\em n}, array2D \& {\em sumkhatqi\_\-iton}, array3D \& {\em sumkhatqikhatqj\_\-itonjton}, long {\em z})\hspace{0.3cm}{\tt  [static]}}\label{qperparraynew_8cpp_d834a8fc7dd96237188220f0ed214a38}




Definition at line 90 of file qperparraynew.cpp.

Referenced by QperpCalculator::CalcQmatrices2().

\begin{Code}\begin{verbatim}92 {
93   for (unsigned long i=0; i<n; ++i)
94   {
95     for (unsigned long j=0; j<n; ++j)
96     {
97       sumkhatqikhatqj_itonjton[z][i][j] = sumkhatqi_iton[z][i] + sumkhatqi_iton[z][j];
98     }
99     sumkhatqikhatqj_itonjton[z][i][n] = sumkhatqi_iton[z][i];
100   }
101   for (unsigned long j=0; j<n; ++j)
102   {
103     sumkhatqikhatqj_itonjton[z][n][j] = sumkhatqi_iton[z][j];
104   }
105   sumkhatqikhatqj_itonjton[z][n][n] = 0.;
106 }
\end{verbatim}
\end{Code}


\index{qperparraynew.cpp@{qperparraynew.cpp}!_FillAllRows1D@{\_\-FillAllRows1D}}
\index{_FillAllRows1D@{\_\-FillAllRows1D}!qperparraynew.cpp@{qperparraynew.cpp}}
\subsubsection{\setlength{\rightskip}{0pt plus 5cm}void @8::\_\-FillAllRows1D (unsigned long {\em n}, array1D \& {\em khatqis}, array2D \& {\em sumkhatqis}, long {\em z})\hspace{0.3cm}{\tt  [static]}}\label{qperparraynew_8cpp_6e28f19c83c2c6e49a2672873048fd10}




Definition at line 75 of file qperparraynew.cpp.

References SwUtils::\_\-NumberToBoolArray().

Referenced by QperpCalculator::CalcQmatrices2().

\begin{Code}\begin{verbatim}76 {
77   // We will be stepping through setting each qi to zero
78   // Create a vector of bools to indicate which qi is zero for this iteration
79   std::vector<bool> zeroed( n );
80 
81   // Convert number z to an array of bools, the ith element saying whether qi should be zero
82   _NumberToBoolArray( z, zeroed, n ); 
83   // Starting point for the iteration: (n-1)th has just one element, khat.q_{n-1}
84   sumkhatqis[z][n-1] = (!zeroed[n-1])*khatqis[n-1];
85   // Now iterate over the rest, adding one khatqi at each step
86   for (unsigned long i=2; i<(n+1); ++i)
87     sumkhatqis[z][n-i] = sumkhatqis[z][n-i+1] + (!zeroed[n-i])*khatqis[n-i];
88 }
\end{verbatim}
\end{Code}


\index{qperparraynew.cpp@{qperparraynew.cpp}!_FillAllRows2DSymmetric@{\_\-FillAllRows2DSymmetric}}
\index{_FillAllRows2DSymmetric@{\_\-FillAllRows2DSymmetric}!qperparraynew.cpp@{qperparraynew.cpp}}
\subsubsection{\setlength{\rightskip}{0pt plus 5cm}void @8::\_\-FillAllRows2DSymmetric (unsigned long {\em n}, array2D \& {\em qiqjs}, array3D \& {\em sumqiqjs}, long {\em z})\hspace{0.3cm}{\tt  [static]}}\label{qperparraynew_8cpp_e6e4ce05ca5129920fbf7d9ef0a1e88f}




Definition at line 66 of file qperparraynew.cpp.

References \_\-FillFirstRow(), \_\-FillNextRow(), and SwUtils::\_\-NumberToBoolArray().

Referenced by QperpCalculator::CalcQmatrices2().

\begin{Code}\begin{verbatim}67 {
68   std::vector<bool> zeroed( n );
69   _NumberToBoolArray( z, zeroed, n ); 
70   _FillFirstRow( n, z, zeroed, qiqjs, sumqiqjs );
71   for (unsigned long i=2; i<(n+1); ++i)
72     _FillNextRow( n, n-i, z, zeroed, qiqjs, sumqiqjs );
73 }
\end{verbatim}
\end{Code}


\index{qperparraynew.cpp@{qperparraynew.cpp}!_FillFirstRow@{\_\-FillFirstRow}}
\index{_FillFirstRow@{\_\-FillFirstRow}!qperparraynew.cpp@{qperparraynew.cpp}}
\subsubsection{\setlength{\rightskip}{0pt plus 5cm}void @8::\_\-FillFirstRow (unsigned long {\em n}, unsigned long {\em z}, std::vector$<$ bool $>$ {\em zeroed}, array2D \& {\em qiqjs}, array3D \& {\em sumqiqjs})\hspace{0.3cm}{\tt  [static]}}\label{qperparraynew_8cpp_2038a405262755765309ef166e170a95}




Definition at line 24 of file qperparraynew.cpp.

Referenced by \_\-FillAllRows2DSymmetric().

\begin{Code}\begin{verbatim}26 {
27   // If the (n-1)th row is zeroed, jump ahead to the answer
28   if (zeroed[n-1])
29   {
30     for (unsigned long i=0; i<n; ++i)
31     {
32       sumqiqjs[z][n-1][i] = 0.; sumqiqjs[z][i][n-1] = 0.;
33     }
34   }
35   else
36   {
37     // The bottom right hand corner of the matrix has just one contribution
38     sumqiqjs[z][n-1][n-1] = qiqjs[n-1][n-1];
39 
40     // Now fill in the lower row and the right most column
41     for (unsigned long i=2; i<(n+1); ++i)
42     {
43       sumqiqjs[z][n-1][n-i] = sumqiqjs[z][n-1][n-i+1] + (!zeroed[n-i])*qiqjs[n-1][n-i];
44       sumqiqjs[z][n-i][n-1] = sumqiqjs[z][n-1][n-i];
45     }
46   }
47 }
\end{verbatim}
\end{Code}


\index{qperparraynew.cpp@{qperparraynew.cpp}!_FillNextRow@{\_\-FillNextRow}}
\index{_FillNextRow@{\_\-FillNextRow}!qperparraynew.cpp@{qperparraynew.cpp}}
\subsubsection{\setlength{\rightskip}{0pt plus 5cm}void @8::\_\-FillNextRow (unsigned long {\em n}, unsigned long {\em i}, unsigned long {\em z}, std::vector$<$ bool $>$ {\em zeroed}, array2D \& {\em qiqjs}, array3D \& {\em sumqiqjs})\hspace{0.3cm}{\tt  [static]}}\label{qperparraynew_8cpp_fa7bacdc3db3da4a62517d2518826fbc}




Definition at line 51 of file qperparraynew.cpp.

Referenced by \_\-FillAllRows2DSymmetric().

\begin{Code}\begin{verbatim}53 {
54   for (unsigned long j=0; j<(i+1); ++j)
55     sumqiqjs[z][i][i-j] = sumqiqjs[z][i+1][i-j] 
56         + sumqiqjs[z][i][i-j+1] - sumqiqjs[z][i+1][i-j+1] 
57         + (!zeroed[i])*(!zeroed[i-j])*qiqjs[i][i-j];
58 
59   for (unsigned long j=1; j<(i+1); ++j)
60     sumqiqjs[z][i-j][i] = sumqiqjs[z][i][i-j];
61 }
\end{verbatim}
\end{Code}


\index{qperparraynew.cpp@{qperparraynew.cpp}!_Iterate1DForZeroes@{\_\-Iterate1DForZeroes}}
\index{_Iterate1DForZeroes@{\_\-Iterate1DForZeroes}!qperparraynew.cpp@{qperparraynew.cpp}}
\subsubsection{\setlength{\rightskip}{0pt plus 5cm}void @8::\_\-Iterate1DForZeroes (unsigned long {\em n}, array1D \& {\em khatqis}, array2D \& {\em sumkhatqis})\hspace{0.3cm}{\tt  [static]}}\label{qperparraynew_8cpp_9b9808a1bf16458a7ebdf81287337ec5}




Definition at line 331 of file qperparraynew.cpp.

References SwUtils::\_\-NumberToBoolArray().

Referenced by QperpCalculator::CalcQmatrices().

\begin{Code}\begin{verbatim}332 {
333   std::vector<bool> zeroed(n);
334   double tmp;
335   for (unsigned long z=1; z<pow(2,n); ++z)
336   {
337     // First, calculate the array of zeroed bools
338     _NumberToBoolArray( z, zeroed, n );
339     // Now, iterated over all the elements
340     for (unsigned long i=0; i<n; ++i) 
341     { 
342       // Now deal with k part
343       tmp = sumkhatqis[0][i];
344       for (unsigned long ii=i; ii<n; ++ii)
345       {
346         if ( zeroed[ii] )
347         {
348           tmp -= khatqis[ii];
349         }
350       }
351       sumkhatqis[z][i] = tmp;
352     }
353   }
354 }
\end{verbatim}
\end{Code}


\index{qperparraynew.cpp@{qperparraynew.cpp}!_Iterate2DForZeroes@{\_\-Iterate2DForZeroes}}
\index{_Iterate2DForZeroes@{\_\-Iterate2DForZeroes}!qperparraynew.cpp@{qperparraynew.cpp}}
\subsubsection{\setlength{\rightskip}{0pt plus 5cm}void @8::\_\-Iterate2DForZeroes (unsigned long {\em n}, array2D \& {\em qiqjs}, array3D \& {\em sumqiqjs})\hspace{0.3cm}{\tt  [static]}}\label{qperparraynew_8cpp_5e634626e89e347ac7105f3a34a7f065}




Definition at line 301 of file qperparraynew.cpp.

References SwUtils::\_\-NumberToBoolArray().

Referenced by QperpCalculator::CalcQmatrices().

\begin{Code}\begin{verbatim}302 {
303   // Now we go through all the possible combinations of zeroed columns and rows
304   std::vector<bool> zeroed( n );
305   double tmp;
306   for (unsigned long z=1; z<pow(2,n); ++z)
307   {
308     // First, calculate the array of zeroed bools
309     _NumberToBoolArray( z, zeroed, n );
310     // Now, iterated over all the elements
311     for (unsigned long i=0; i<n; ++i) 
312     { 
313       for (unsigned long j=0; j<n; ++j) 
314       {
315         // If either the row or column has been zeroed, subtract off from answer
316         tmp = sumqiqjs[0][i][j];
317         for (unsigned long ii=i; ii<n; ii++)
318         {
319           for (unsigned long jj=j; jj<n; jj++)
320           {
321             if ( zeroed[ii] || zeroed[jj] )
322               tmp -= qiqjs[ii][jj];
323           }
324         }
325         sumqiqjs[z][i][j] = tmp;
326       }
327     }
328   }
329 }
\end{verbatim}
\end{Code}


\index{qperparraynew.cpp@{qperparraynew.cpp}!_OuterProduct0D1D@{\_\-OuterProduct0D1D}}
\index{_OuterProduct0D1D@{\_\-OuterProduct0D1D}!qperparraynew.cpp@{qperparraynew.cpp}}
\subsubsection{\setlength{\rightskip}{0pt plus 5cm}void @8::\_\-OuterProduct0D1D (unsigned long {\em n}, double {\em mag1}, double {\em th1}, MyArray \& {\em mags2}, MyArray \& {\em ths2}, array1D \& {\em res})\hspace{0.3cm}{\tt  [static]}}\label{qperparraynew_8cpp_fc899dbd2fbe818860b4c384c52fb84f}




Definition at line 293 of file qperparraynew.cpp.

Referenced by QperpCalculator::CalcQmatrices().

\begin{Code}\begin{verbatim}294 {
295   for (unsigned long i=0; i!=n; ++i)
296     res[i] = mag1 * mags2[i] * cos( th1 - ths2[i] );
297 }
\end{verbatim}
\end{Code}


\index{qperparraynew.cpp@{qperparraynew.cpp}!_OuterProduct1D1D@{\_\-OuterProduct1D1D}}
\index{_OuterProduct1D1D@{\_\-OuterProduct1D1D}!qperparraynew.cpp@{qperparraynew.cpp}}
\subsubsection{\setlength{\rightskip}{0pt plus 5cm}void @8::\_\-OuterProduct1D1D (unsigned long {\em n}, MyArray \& {\em mags1}, MyArray \& {\em ths1}, MyArray \& {\em mags2}, MyArray \& {\em ths2}, array2D \& {\em res})\hspace{0.3cm}{\tt  [static]}}\label{qperparraynew_8cpp_7d756d821e729c3031739ae9168483e4}




Definition at line 280 of file qperparraynew.cpp.

Referenced by QperpCalculator::CalcQmatrices().

\begin{Code}\begin{verbatim}281 {
282   for (unsigned long i=0; i!=n; ++i)
283     for (unsigned long j=i+1; j!=n; ++j)
284     {
285       res[i][j] = mags1[i] * mags2[j] * cos( ths1[i] - ths2[j] );
286       res[j][i] = res[i][j];
287     }
288   
289   for (unsigned long i=0; i!=n; ++i)
290     res[i][i] = mags1[i] * mags2[i];
291 }
\end{verbatim}
\end{Code}


