\section{Gyulassy/opacity3/src/main\_\-qperptest.cpp File Reference}
\label{main__qperptest_8cpp}\index{Gyulassy/opacity3/src/main_qperptest.cpp@{Gyulassy/opacity3/src/main\_\-qperptest.cpp}}
{\tt \#include $<$iostream$>$}\par
{\tt \#include $<$string$>$}\par
{\tt \#include $<$boost/array.hpp$>$}\par
{\tt \#include \char`\"{}timer.h\char`\"{}}\par
{\tt \#include \char`\"{}./glv3/qperpcalculator1.h\char`\"{}}\par
{\tt \#include \char`\"{}./glv3/qperpcalculator3.h\char`\"{}}\par
\subsection*{Functions}
\begin{CompactItemize}
\item 
int {\bf main} (int argc, char $\ast$argv[$\,$])
\end{CompactItemize}


\subsection{Function Documentation}
\index{main_qperptest.cpp@{main\_\-qperptest.cpp}!main@{main}}
\index{main@{main}!main_qperptest.cpp@{main\_\-qperptest.cpp}}
\subsubsection{\setlength{\rightskip}{0pt plus 5cm}int main (int {\em argc}, char $\ast$ {\em argv}[$\,$])}\label{main__qperptest_8cpp_0ddf1224851353fc92bfbff6f499fa97}




Definition at line 15 of file main\_\-qperptest.cpp.

References Timer::DisplayResult(), QperpCalculator3$<$ n $>$::GetSumQs1k(), QperpCalculator1$<$ n $>$::GetSumQs1k(), QperpCalculator3$<$ n $>$::GetSumQskk(), QperpCalculator1$<$ n $>$::GetSumQskk(), QperpCalculator3$<$ n $>$::SetK(), QperpCalculator1$<$ n $>$::SetK(), QperpCalculator3$<$ n $>$::SetQsThetas(), QperpCalculator1$<$ n $>$::SetQsThetas(), Timer::StartTimer(), and Timer::StopTimer().

\begin{Code}\begin{verbatim}16 {
17   std::cout << "QperpTester" << std::endl;
18 
19   boost::array<double, 4> qs; qs[0] = 1; qs[1] = 2; qs[2] = 3; qs[3] = 4;
20   boost::array<double, 4> ths; ths[0] = 0; ths[1] = 0; ths[2] = 0; ths[3] = 0;
21 
22   QperpCalculator1<4> qp1; QperpCalculator3<4> qp2;
23   qp1.SetQsThetas( qs, ths ); qp2.SetQsThetas( qs, ths );
24   qp1.SetK( 1. ); qp2.SetK( 1. );
25 
26   for (long z=0; z<=7; ++z)
27   {
28     std::cout << "Qskk[" << z << "]" << std::endl;
29     for (long i=1; i<=5; ++i)
30     {
31       if ( qp1.GetSumQskk( i, z ) != qp2.GetSumQskk( i, z ) )
32       {
33         std::cout << "(i,z)=(" << i << "," << z << ") ";
34         std::cout << "1: " << qp1.GetSumQskk( i, z ) << ", 2: " << qp2.GetSumQskk( i, z ) << std::endl;
35       }
36     }
37     std::cout << std::endl;
38 
39     std::cout << "Qs1k[" << z << "]" << std::endl;
40     for (long i=1; i<=5; ++i)
41     {
42       if ( qp1.GetSumQs1k( i, z ) != qp2.GetSumQs1k( i, z ) )
43       {
44         std::cout << "(i,z)=(" << i << "," << z << ") ";
45         std::cout << "1: " << qp1.GetSumQs1k( i, z ) << ", 2: " << qp2.GetSumQs1k( i, z ) << std::endl;
46       }
47     }
48     std::cout << std::endl << std::endl;
49   }
50 
51 
52   QperpCalculator1<1> qp_v1_1; QperpCalculator3<1> qp_v2_1;
53   QperpCalculator1<10> qp_v1_10; QperpCalculator3<10> qp_v2_10;
54   Timer timer;
55   unsigned long num;
56 
57   boost::array<double, 1> qs1, ths1;
58   qs1[0] = 1.; ths1[0] = 0.23;
59   num = 500000;
60 
61   std::cout << "Version 1 (old): " << num << " at n=1: " << std::endl;
62   timer.StartTimer();
63   for (unsigned long i=0; i!=num; ++i)
64     qp_v1_1.SetQsThetas(qs1, ths1);
65   timer.StopTimer(); timer.DisplayResult(1); timer.DisplayResult(num);
66   std::cout << std::endl;
67 
68   std::cout << "Version 2 (new): " << num << " at n=1: " << std::endl;
69   timer.StartTimer();
70   for (unsigned long i=0; i!=num; ++i)
71     qp_v2_1.SetQsThetas(qs1, ths1);
72   timer.StopTimer(); timer.DisplayResult(1); timer.DisplayResult(num);
73   std::cout << std::endl;
74 
75   boost::array<double, 10> qs10, ths10;
76   for (long i=0; i<10; ++i)
77   {
78     qs10[i] = 2.; ths10[i] = 1.;
79   }
80   num = 100;
81 
82   std::cout << "Version 1 (old): " << num << " at n=10: " << std::endl;
83   timer.StartTimer();
84   for (unsigned long i=0; i!=num; ++i)
85     qp_v1_10.SetQsThetas(qs10, ths10);
86   timer.StopTimer(); timer.DisplayResult(1); timer.DisplayResult(num);
87   std::cout << std::endl;
88 
89   std::cout << "Version 2 (new): " << num << " at n=10: " << std::endl;
90   timer.StartTimer();
91   for (unsigned long i=0; i!=num; ++i)
92     qp_v2_10.SetQsThetas(qs10, ths10);
93   timer.StopTimer(); timer.DisplayResult(1); timer.DisplayResult(num);
94   std::cout << std::endl;
95 
96 /*
97 
98   long NumberOfIterations = 10000;
99   std::string resume("n");
100   std::string inputFile = "./temp.params";
101   std::string outputFile = "./temp.txt";
102   
103   Driver<GlvRadiative1<9>, Store2D, 9> myDriver;
104   //Driver<RadCalcerWrapper<9>, Store2D, 9> myDriver;
105   myDriver.Setup( resume, inputFile );
106 
107   Timer timer;
108 
109   // Set up a progress bar for feedback
110   // The first number of the current point (start = 0)
111   // The second is the end point, the total number
112   ProgressBar progBar(0,NumberOfIterations);
113   progBar.PrintPreliminaries();
114   
115   // And now we run the main loop
116   timer.StartTimer();
117   for (long i=0; i<NumberOfIterations; ++i)
118   {
119     progBar.SetNow(i);
120     progBar.PrintProgress();
121     myDriver.RunOneIteration();
122   }
123   progBar.SetNow(NumberOfIterations);
124   progBar.PrintProgress();
125   timer.StopTimer();
126   // And the main loop is done!
127 
128   myDriver.SaveResults( outputFile );
129  
130   // Print to screen a few details of the run
131   progBar.PrintFinal();
132   double totTime = timer.GetResult();
133   std::cout << "Total time = " << totTime << " seconds\n";
134   std::cout << "Time per iteration = " << totTime/static_cast<double>(NumberOfIterations);
135   std::cout << "\n";
136   std::cout << std::endl;
137 
138 */
139 
140   return EXIT_SUCCESS;
141 }
\end{verbatim}
\end{Code}


